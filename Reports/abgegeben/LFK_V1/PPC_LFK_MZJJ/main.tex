
\documentclass{classes/report}

\usepackage{blindtext}
\usepackage{titlesec}

\setcounter{secnumdepth}{3}

%\title{Determining Molar Enthalpy and Entropy of Vaporization for Acetone and n-Hexane}
\title{Measuring conductivity of water, sodium carbonate and potassium carbonate solutions under varying conditions}

\assistant{Toutoudaki Eirini}{eirini.toutoudaki@phys.chem.ethz.ch}

\authors{Janosch Jörg}{jjoerg@ethz.ch}{DCHAB}{Maria Zimmermann}{mazimmerm@ethz.ch}{DCHAB}

\abstract{}

\begin{document}

    \maketitle
    \newpage
    
    \begin{multicols}{2}
    
        \section{Introduction}
    
            \input{parts/introduction.tex}
            
        %\newpage    
        
        
            \section{Experimental}

%Characterize the system, give all details that matter. Describe how the experimental procedure went.

% CAS codes Hinzufügen?

\subsection{Chemicals}
The following experiments were conducted using sodium carbonate, anhydrous, (\qty{105.99}{\gram\per\mole}), from \texttt{Sigma Aldrich}, puriss., (\qty{99.5}{\percent}, calc. to the dried substance), and potassium carbonate (\qty{138.2}{\gram\per\mole}), from \texttt{Sigma Aldrich}, (\qty{99}{\percent}). All chemicals were used forgoing any further purification.

Additionally, a premixed $\approx$ \qty{0.1}{\M} calcium chloride solution was used. Its origin and purity are unknown to the authors.

\subsection{Procedure}
A total of 5 experiments were prepared and carried out as follows:

\begin{figure}[H]
    \centering
    \includegraphics[width=.5\textwidth]{figures/Conductometer.pdf}
    \caption{The \texttt{KNICK SE 204} measuring cell can measure values ranging from \qty{1.00}{\micro\siemens\per\centi\meter} - \qty{500}{\milli\siemens\per\centi\meter} and has a cell constant of approximately \qty[per-mode=reciprocal]{0.5}{\per\centi\meter}}
    \label{fig:sketch_cond}
\end{figure}






\subsubsection{Specific conductivity of water}

First, the specific conductivity $\kappa$ of differently treated water samples was measured using a \texttt{KNICK 703} conductometer with a \texttt{KNICK SE 204} 4-pole measuring cell (Fig. \ref{fig:sketch_cond}). Three samples were prepared for deionised water, ultrapure deionised and degassed water (from a \texttt{HUBER and CO. AG} TKA-GenPure), and tap water each and the respective conductivity measured immediately after preparation (especially for deionised and degassed water) to ensure that the results would not be falsified by any compounds formed through contact with the surrounding air. 


%\newpage

\begin{figure}[H]
    \centering
    \includegraphics[width=.5\textwidth]{figures/Conductometer_bigpicture.pdf}
    \caption{The entire setup for conductivity measurements, including the \texttt{LAUDA MT} circulation thermostat, the \texttt{KNICK SE 204} measuring cell and the \texttt{KNICK Conductometer 703}.}
    \label{fig:sketch_cond_2}
\end{figure}

\subsubsection{Correlation between temperature and conductivity}

For the next experiment, the specific conductivity $\kappa$ of a 0.1 M potassium carbonate solution was measured at steadily decreasing temperature. 
The solution was prepared by weighing \qty[round-precision=5]{1.3895}{\gram} of sodium carbonate in a \qty{100}{\milli\liter} volumetric flask using a \texttt{METTLER TOLEDO AG204 Delta Range} analytical balance and filling the flask to its mark with deionized water. 
The sample was transferred to the measuring vessel (Fig. \ref{fig:sketch_cond_2}) and heated to \qty{50}{\celsius} by means of a \texttt{LAUDA MT} circulation thermostat. The specific conductivity at \qty{50}{\celsius} was noted in the lab journal and subsequently the heat supply to the circulation thermostat cut off, simultaneously activating the water cooler to bring the temperature of the sample to a gradual, steady decline. 
Finally, $\kappa$ of potassium carbonate in a 0.1 M aqueous solution was noted down for temperatures between \qty{50} {\celsius} and \qty{25}{\celsius} in steps of \qty{1}{\celsius}.



\subsubsection{Molar conductivity of different electrolytes} 

Next, the molar conductivity $\Lambda$ of \qty{0.01}{\M} solutions of potassium carbonate and sodium carbonate was calculated using values obtained from measurements of $\kappa$ at constant temperature for both solutions. 
The potassium carbonate solution was prepared by taking \qty{50}{\milli\liter} of the solution used in the previous experiment and diluting it with water at a 1:10 ratio. 
The sodium carbonate solution was prepared similarly to the potassium carbonate solution in the previous experiment, by weighing \qty{0.5298}{\gram} of sodium carbonate in a \qty{100}{\milli\liter} volumetric flask and filling it to its mark.
Samples of both solutions were transferred to the measuring vessel separately and their respective conductivity noted down after brief stirring. This was repeated twice more for each solution.

\subsubsection{Concentration dependency of conductivity and limiting molar conductivity}

For the next experiment, the specific conductivity $\kappa$ of a \qty{0.01}{\M} sodium carbonate solution was measured at constant temperature and increasing concentration $c$. 
First, a \qty{10}{\milli\liter} burette was rinsed out and then filled with the same solution as in the previous experiment. Then the measuring vessel was filled with \qty{100}{\milli\liter} of deionized water (measured by means of a volumetric flask) and its intrinsic conductivity documented. Finally, the sodium carbonate solution was added in steps of \qty{0.5}{\milli\liter} and the conductivity of the resulting solution noted down after each increase until a total of \qty{10}{\milli\liter} had been added. 

\subsubsection{Conductometric titration}

For the last experiment, a potassium carbonate solution of unknown concentration was provided by the laboratory assistant, of which exactly \qty{100}{\milli\liter} were transferred to the measuring vessel via volumetric flask. A \texttt{METROHM 775} dosimat was filled with the titer, for which a pre-prepared \qty[round-precision=5]{0.09997}{\M} calcium chloride solution was provided. The titer was released into the potassium carbonate solution via "Go" button in approximately \qty{0.5}{\milli\liter} steps, noting down the exact volume that was added and the corresponding concentration after every increment. A total of \qty{20}{\milli\liters} calcium chloride was added to the solution. 





% ===============================================================
% ==================OLD STUFF====================================
% ===============================================================

\iffalse{


\subsubsection{measuring conductivity of different solutions}

measuring conductivity of differnt types of water, versch elektrolytlösungen, versch. salze in wasser,  eines salzes bei 

wie funktioniert conductometer (KNICK 703 und 4-Pol-Messzelle KNICK SE 204 
integrierter NTC Temperatursensoror

kann gleichzeitig 

0.09997 M cacl2







The identity and purity of acetone and n-hexane were verified by determining their respective \textit{refractive index} and \textit{density}. 






\textit{Refractive index} measurements were done via digital refractometer \texttt{ATAGO RX-5000} operating at standard wavelength of \qty{589.0}{\nano\meter} (D-line) and with samples maintained at a constant \qty{20.00 \pm 0.02}{\celsius} by a \texttt{LAUDA E100} circulation thermostat (attached to the refractometer). After ensuring the glass surface of the sample block was dry, a small amount of liquid was applied – just enough to cover the circular surface – and once equilibration was reached as indicated by the temperature display, measurements could be initiated.

\textit{Density} was determined in two different ways.

First, by \textit{method A}, filling a \qty{25.00 \pm 0.04}{\milli\liter} volumetric flask (insert marke) with either liquid and measuring its mass using a \texttt{METTLER-TOLEDO AG204 Delta Range} analytical balance, the accuracy range of which is stated as \qty{0.1}{\milli\gram} by the manufacturer.

And secondly, by \textit{method B}, using an \texttt{ANTON PAAR DMA 48} density gauge (Fig. \ref{fig:sketch_rho}), where the respective sample was carefully inserted via plastic syringe and – after visual confirmation that there were no air bubbles inside the u-shaped glass tube – run at setting \mintinline{R}{F505}. Between measurements of different samples, the tube was rinsed with deionized water and dried by inducing airflow.





The boiling temperatures of acetone and n-hexane were measured at different pressure settings. 

For this purpose, a two necked flask containing the sample and about 10-15 boiling stones and filled to approximately half its capacity was attached to a pre-prepared setup (Fig.\ref{fig:sketch_setup}), omitting grinding grease due to the risk of contaminating the sample and comparatively low significance of a tight seal in this experiment.

Only after ensuring that all openings were closed, the \texttt{BÜCHI VAC V-503} vacuum pump, dimroth condensation cooler and \texttt{WINKLER WHLG2} laboratory heating mantle with a \texttt{WL10} heating controller were turned on. It is important to note that forgoing the former here and starting the cooler before sealing the system would allow water vapor from the air to condense inside the cooler and likely lead to falsified results. 

While maintaining a low heat supply via the controllable heating mantle, pressure within the system was steadily decreased under close surveillance through evacuation by means of a \texttt{BÜCHI I-100} vacuum controller and a ventilation valve until about \qty{150}{\milli\bar} and \qty{100}{\milli\bar} were reached for acetone and n-hexane respectively and the liquid was simultaneously boiling inside the flask and dripping steadily from the cooler. For acetone we could not observe any dripping at first, yet the liquid seemed to be boiling and the temperature was quickly decreasing to under \qty{10}{\celsius}. We concluded that the vapor currently forming would be at a lower temperature than the liquid in the dimroth cooler, causing it to escape into the separator instead of condensing. As such we increased our heat supply and not long after, equilibrium returned to the system and steady dripping could be observed. 

The temperature measured by the \texttt{GREISINGER GMH 3210} digital temperature gauge (with a resolution of \qty{0.1}{\kelvin}) was marked down with its corresponding vapor pressure and the pressure was incremented in \qtyrange{25}{50}{\mbar} steps until reaching \qty{900}{\mbar}, waiting for the temperature to level out after each increase and recording respective values. This was repeated at decreasing pressure, generating two sets of values for each sample.

%\newpage






The evaporation cooling effect of acetone, n-hexane and methanol (reference sample) were visualized and recorded by measuring the surface temperature of an ultra-sensitive heat sensor at \qty{0.25}{\second} intervals during the evaporation of a predefined volume of liquid sample applied to the sensor.

The setup, a \texttt{TREVAC}-apparatus (transient evaporation cooling) as can be seen in Fig. \ref{fig:sketch_trevac}, self-developed by the \texttt{PCL} at \texttt{ETH Zürich}, relies on a \texttt{LAUDA Ecoline 103} thermostat to keep the aluminium block with the \texttt{NATIONAL LM 35} heat sensor at its center at a programmable base temperature of 

$T_0=$ \qty{35}{\celsius}, 
\\deviating less than \qty{\pm 0.01}{\kelvin}. $T_0$ was selected at \qtyrange{30}{40}{\kelvin} below the boiling point of the lowest-boiling liquid. The sample was inserted at a steady pace via a \texttt{HAMILTON 801 RN} microliter syringe (scale facing forward to ensure measurement circumstances were as similar as possible for each sample inserted), using a measuring gauge to measure out exactly \qty{5}{\micro\liter} and discard any excess liquid beforehand. 

The registered surface temperature over time was recorded via analog/digital converter (ADC, 23 bit) and broadcast on the display of a laptop attached to the setup. 

This process was repeated 3 times per substance, leaving enough time for the sensor to equilibrate back to $T_0$ between the end of each prior evaporation and insertion of the current sample.




\subsection{}

}
        
            %\newpage

\section{Results and Discussion}
The analysis and calculations were conducted using the R programming language \cite{R} and Python Jupyter Notebooks \cite{IPython:2007}. The scripts are included in the appendix of this report. \ref{r_scripts} All uncertainties stated are provided for a 95 \% confidence interval.


\subsection{Density and refractive index}

The refractive index and density of acetone were measured at 

$n_D^{20} =$ \qty{1.35867}{1} and 

$\rho = $ \qty{0.7913}{\density} (\textit{method B}) 
\\and the density of acetone was calculated (\textit{method A}) to be 

\qty{0.7904 \pm 0.0026}{\density} 
\\corresponding with literature values of 

$n_D^{20} =$ \qty{1.3588}{1} and 

$\rho = $ \qty{0.7899}{\density}. \cite{meister} 
\\For n-hexane values of 

$n_D^{20} =$ \qty{1.37506}{1} and 

$\rho = $ \qty{0.6594}{\density} (\textit{method B}) and 

$\rho = $ \qty{0.6572 \pm 0.0022}{\density} (\textit{method A}) 
\\were obtained compared to literature values of 

$n_D^{20} =$ \qty{1.3751}{1} and 

$\rho = $ \qty{0.6603}{\density}. \cite{meister} 
\\It can be observed, that the results obtained for acetone are more precise than those of n-hexane. This could be explained for example by the varying grades of purity (99.5\% vs. 95\% specified by their producers.
\begin{figure}[H]
    \centering
    \includegraphics[width=.5\textwidth]{figures/rho-comparison.pdf}
    \caption{Comparisons of the density values measured, and reference values found in literature \cite{meister}.}
    \label{fig:rho_comp}
\end{figure}


\subsection{Vapor Pressure}

\begin{figure}[H]
    \centering
    \includegraphics[width=.5\textwidth]{figures/DDR1_t_p.pdf}
    \caption{The vapor pressure curves illustrate the exponential correlation between vapor pressure and temperature. (Standard boiling points extrapolated and marked by the vertical and horizontal lines.)}
    \label{fig:ddr1_t_p}
\end{figure}

As mentioned in the introduction, vapor pressure can be described as $p(T)$. Accordingly, the results obtained from measuring the samples' boiling temperatures at differing surrounding pressures were plotted in a $p$-$T$-diagram, presented in Fig. \ref{fig:ddr1_t_p}.

Enthalpies of vaporization were calculated with by plotting logarithmic $\frac{p}{p_0}$ against inverse temperature $T$ via a linear regression model (Fig. \ref{fig:ddr1_inv_ln}) and calculating the slope $b$ of the resulting linear graphs. Expansion of equation (\ref{eq:6.2}) shows that it is equal to $\frac{\Delta_VH}{R}$. Results amounted to 

$\Delta_VH=$ \qty{32.49 \pm 0.26}{\kJpmole} for acetone and 

$\Delta_VH=$ \qty{32.66 \pm 0.25}{\kJpmole} for n-hexane. 
\\Acetone is a polar solvent and n-hexane is not. Because of this, one would assume, that its enthalpy of vaporization and standard boiling temperature would be higher than those of n-hexane because of stronger intermolecular bonds. Evidently, this is not the case. A possible explanation could be that though it is polar, acetone has a lower molar mass than n-hexane, meaning it would take less  energy to bring one mol of acetone from its fluid to its gaseous form than one mol of another substance with similar polarity.

 
\begin{figure}[H]
    \centering
    \includegraphics[width=.5\textwidth]{figures/DDR1_inv_ln.pdf}
    \caption{The Enthalpy of vaporization for each sample can be visualized in a $\ln\left(\frac{p}{p_0}\right)$ - $\frac{1}{T}$ diagram, because it is directly proportionate to the linear graph's slope $b$ (by factor of the gas constant $R$.)}
    \label{fig:ddr1_inv_ln}
\end{figure}


Normal boiling temperatures $T_0$ and standard entropies $\Delta_VS(T_0)$ of vaporization were also calculated with the help of the linear model parameters intercept $a$ and slope $b$ and found to be 

$T_0^{\text{acet}}$ = \qty{56.29}{°C}  

$T_0^{\text{nhex}}$ = \qty{68.06}{°C}
\\corresponding to literature values of 

$T_0^{\text{acet}}$ = \qty{56.15 \pm 0.3}{°C} 

$T_0^{\text{nhex}}$ = \qty{68.75 \pm 0.3}{°C}

and

$\Delta_VS(T_0)^{\text{acet}}$ = \qty{98.62}{\joule\per\mole\per\kelvin}

$\Delta_VS(T_0)^{\text{nhex}}$ = \qty{95.74}{\joule\per\mole\per\kelvin}.





\subsection{Transient evaporation cooling}
The data files generated by the \texttt{TREVAC} apparatus were imported to R. With the \mintinline{R}{identify()} command, the relevant time ranges for the peak integration could be selected graphically. (Fig. \ref{fig:nhex_peaks}) This data, including statistical parameters were then exported to a \textit{csv} file for documentation purposes and for further processing. To calculate the final values, including standard errors, a Jupyter notebook \cite{IPython:2007} with the \texttt{METAS UncLib} uncertainty modeling software \cite{unclib} was used. Formula \ref{eq:6.18} was 

For acetone, \qty{30.2 \pm 2.8}{\kJpmole} was calculated, and for n-hexane, \qty{32.0 \pm 1.9}{\kJpmole}, was found.

\begin{figure}[H]
    \centering
    \includegraphics[width=.5\textwidth]{figures/n-hexane.pdf}
    \caption{Temperature peaks caused by n-hexane withdrawing heat from the measurement surface. The grey area is directly correlating with the enthalpy of evaporation $\Delta_VH_{\text{subst}}$ of the substance.}
    \label{fig:nhex_peaks}
\end{figure}

Standard values for $\Delta_VH$ can be found on the website of the \texttt{National Institute of Standard Technology}:

$\Delta_VH^0_{\text{acet}}$ = \qty{31.27}{\kJpmole} \cite{NIST:acet}

$\Delta_VH^0_{\text{nhex}}$ = \qty{31\pm 1}{\kJpmole} \cite{NIST:nhex}



            \section{Coursebook exercises}

\subsection{Ion velocity}

Using equation \ref{eq:7.4}:

\begin{equation} \label{eq:7.4.ex}
    v_i = \frac{e E}{6 \pi \eta} \frac{z_i}{r_i}
\end{equation}

yields $v_{\mathrm{Mg^{2+}}} =$ \qty{1.70}{\micro\meter\per\second} and $v_{\mathrm{CH_3COO^{-}}} =$ \qty{0.85}{\micro\meter\per\second}.

$\mathrm{Mg^{2+}}$ moves faster than $\mathrm{CH_3COO^{-}}$ due to its higher charge (and smaller size, which is not even taken into account in this exercise). A speed of micrometers per second seems small at first glance, but is really fast compared to the ions size and the density of particles in a liquid.




\subsection{Molar conductivity of pure water}

$$\Lambda \approx \Lambda^0 = \lambda_+^0 + \lambda_-^0 = \qty{547}{\Smolar}$$

$$pH = 7 \implies [H^+] = \qty{1e-7}{\mole\per\liter}$$

$$= \qty{1e-10}{\mole\per\centi\meter\cubed}$$

$$\kappa = \Lambda c = \qty{547}{\Smolar}\qty{1e-10}{\mole\per\centi\meter\cubed}$$

$$= \qty{5.47e-8}{\siemens\per\centi\meter}$$



\subsection{Spagetthi water}

$$\qty{8}{\gram} \text{ NaCl in water } \widehat{=} 0.137 \text{ } \unit{\mole\per\liter} = \qty{137}{\mole\per\centi\meter\cubed}$$

$$\Lambda \approx \Lambda^0 = \lambda_+^0 + \lambda_-^0 = \qty{127}{\Smolar}$$

$$\kappa = \Lambda c = \qty{127}{\Smolar}\qty{137}{\mole\per\centi\meter\cubed}$$

$$= \qty{0.928}{\siemens\per\centi\meter}$$



\subsection{Limiting molar conductivity HCl and KOH}

$$\Lambda^0_{\mathrm{HCl}} = \lambda_{\mathrm{H^+}}^0 + \lambda_{\mathrm{Cl^-}}^0 = \qty{426}{\Smolar}$$

$$\Lambda^0_{\mathrm{KOH}} = \lambda_{\mathrm{K^+}}^0 + \lambda_{\mathrm{OH^-}}^0 = \qty{271}{\Smolar}$$

The calculated values are slightly higher than those extrapolated in Fig. 7.4 of the coursebook. 




\subsection{Reading the titration curve}

$V_t = \qty{18.30 \pm 0.25}{\milli\liter}$

$\widehat{=} \qty{1.800 \pm 0.025}{\milli\mole}$

$\widehat{=} \qty{0.249 \pm 0.003}{\gram}$

%alles abgecheckt von meiner seite aus :)
% yayyy!! und du hast wieder Zugriff :D


            %hallihal
            %lo, kannst du vlt mal kurz im abstract deinen senf dazugeben hihi
            % top mach ich :) ? und gibts sonst noch etwas, dass ich noch machen könnte??
            % du hast dir glaub mal ne Pause verdient ;) du aber auch also wir machen das jetzt hier gemeinsam fertig :) 
            %du sogar noch mehr ich hab bei dem jetzt so wenig gemacht :( => Du hast die ganze Introduction und experimental super gut geschrieben! UUuuuund die superduperawesomemegawunderschönen Grafiken gezeichnet :)  Ich bin wirklich richtig fan davon. Kann ich dich dann mal anstellen für meine Bachelorarbeit? ;) hahahahahahaah danke :)))) ja klar... aber das ist ja nicht der anspruchsvolle teil gewesen (cry :*(, den hast du gemacht mit dem Auswerten und verstehen und ausrechenn und interpretieren.... wenn ich ehrlich bin, hab ich überhaupt keine ahnung, was unsere results bedeuten hahahhahaha. Tbh gibts da glaub auch nicht so viel tiefere Bedeutung hihi. Wir haben einfach Werte erhalten, die einigermassen mit dem der anderen übereinstimmt, aber doch ziemlich Schwankungen aufzeigt. Aber besser als bei Lilian und Aniko, die waren bei einem Wert um Faktor 10 daneben uuuupsiii. HAHAHAHAHAA, das wär mir saaaaafe auch passiert  ironischerweise hat es für einen der Werte (glaub K2CO3) gestummen, aber für NaCO3 nicht :clown-face: lol haha bzw. nicht-lol hahaha  egalllllllllllll alles nicht so wichtig :) 
        
        %YOLOOOOO hahahahahaha ja
        % alors, ich geh mir jetzt schnell Wasser holen, dann schau ich mir das Abstract an. Ich habe übrigens noch eine kleine Änderung gemacht in der Introduction. Siehe Kommentar :)
        % Der Rest sollte soweit stehen, abgesehen von den Übungen. Die Kap. Introduction/Experimental/Results/Discussion hatte ich vorhin nochmals durchgekämmt, sind 96% perfekt ;) yayyyyyyyy, ich hab auch noch paar sachen vorhin schon korrigiert, wir sollten zb darauf achten, dass die groß und kleinschreibung in den überschriften einheitlich ist.... ah oke, dann ists perfekt, so hätt ichs auch gemacht, aber hab ein paar große gesehen und dann hab ihs quasi falsch korrigieret
    % ich hatte bei mir jetzt auch alles klein korrigiert, weil gross irgendwie kacke aussieht hahaha -- tu mal neu rendern, dann solltes alles einheitlich sein :sonnenbrille-emoji: => :janosch-geht-wasser-holen-emoji: :zwinker-emoji:

%perfectttttt, wenns oke ist, könntest du da noch die werte einfügen einfach, es ist echt nix tolles, wirklich das bare minimum, weil mein hirn ist schon matsch ahha -- ohjee! Dann gönn dir wirklich mal eine kurze Pause. Aufstehen, dehnen, 2x ums Haus rennen (seriously), ...
        % alles eine Frage der Perspektive....
        % für mich ist das Programmieren viel gemütlicher hahahah :)))))) du genie aber deal, bei der bachelorarbeit mach ich dir hammer grafiken und du hilfst mir hammer graphen zu machen hehe, team work makes the dream work <3 <3 <3



        
        % well, ich hatte ja zuvor schon einiges an pause mit rennen, essen, im (kalten!) Brunnen baden gehen... hahaha booooooooah das klingt so nice (schmelzemojiiiiii)
        
        %\section{Conclusion}
        
            %
BLABLA

\iffalse{
The experiment was divided in three parts. The first two consisted of two different methods to determine the vaporization enthalpy and entropy of Cyclohexane and 1-Propanol. The first method was by measuring the boiling temperature at different pressures. For the second method the transient vaporization cooling was measured with the TREVAC-Apparatus provided by the ETH. Both methods turned out to be accurate, however the second one provided more accurate results which must have been due to the better thermometer which had a way smaller uncertainty. \\

The third part was the verification of Cyclohexane and 1-Propanol by measuring their densities and refraction indices. The density was once measured the naive way by weighing a known volume and once with a very expensive density measuring device. The second method delivered very accurate results, whereas the first was just a bit off the literature values but still good. The refraction indices were also measured by a device which directly gave back the correct values.
}
        
        % References
        %\input{References.bib}
        \printbibliography[
            heading=bibintoc,
            title={References}
        ]

        
    \end{multicols}
    \newpage
      
    \appendix
    \section{Appendix}
    
        % \subsection{Corse book exercises}
\includepdf[pages=1, scale = 0.7, pagecommand={\section{Appendix}\subsection{Course book exercises}}]{figures/KAL_exercises.pdf} 

\subsection{Labjournal}

\includegraphics[scale = 0.2]{figures/photo_1.jpg} 
\includegraphics[scale = 0.2]{figures/photo_2.jpg} 
\includegraphics[scale = 0.2]{figures/photo_3.jpg} 
\includegraphics[scale = 0.2]{figures/photo_4.jpg} 
\includegraphics[scale = 0.2]{figures/photo_5.jpg} 




\subsection{R and Python scripts} \label{r_scripts}



\subsubsection{Calibration and specific heat capacity of ethanol/water mixture}
\Rfile{scripts/kal_calib.R}

\includepdf[pages=1, scale = 0.75, pagecommand=\subsubsection{Uncertainty estimate of calibration and specific heat capacity of ethanol/water mixture}]{scripts/uncertainty_modelling_kal_calib.pdf} 
\includepdf[pages=2-4, scale = 0.75,pagecommand={}]{scripts/uncertainty_modelling_kal_calib.pdf} \label{append:unc_calib}


\subsubsection{Specific heat capacity of ethanol - overview plot}
\Rfile{scripts/kal_ethanol_alldata.R}

\newpage


\subsubsection{Solution enthalpy of ammonium nitrate}
\Rfile{scripts/kal_sol.R}

\includepdf[pages=1, scale = 0.75, pagecommand=\subsubsection{Uncertainty estimate of solution enthalpy of ammonium nitrate}]{scripts/kal_sol_enth_results_unc_modelling.pdf} 
\includepdf[pages=2-, scale = 0.75,pagecommand={}]{scripts/kal_sol_enth_results_unc_modelling.pdf} \label{append:unc_sol}



\subsubsection{Calorimetric titration of vinegar}
\Rfile{scripts/kal_vinegar.R}
\Rfile{scripts/kal_vinegar_discontinuous.R}

\begin{figure}[H]
    \centering
    \includegraphics[width=.95\textwidth]{figures/plots/vinegar_uncont_highres.pdf}
    \caption{Discrete titration of vinegar with sodium hydroxide.}
    \label{fig:vinegar_uncont_highres}
\end{figure}

\newpage


\subsubsection{Melting enthalpy of ice}
\Rfile{scripts/kal_ice.R}

\includepdf[pages=1, scale = 0.75, pagecommand=\subsubsection{Uncertainty estimate of melting enthalpy of ice}]{scripts/kal_ice_results_uncert.pdf} 
\includepdf[pages=2-, scale = 0.75,pagecommand={}]{scripts/kal_ice_results_uncert.pdf} \label{append:unc_ice}


\subsubsection{Plotting helper scripts}
\Rfile{scripts/helpers.R}







    
    %\section{Acknowledgment}
        
        %
The authors would like to thank the team at DCHAB for enabling this programme.


\iffalse{
The authors would like to thank the ETH Zürich for providing them with the measuring infrastructure and support to be able to conduct the experiments. They would also like to thank Dr. Erich Christian Meister for spontaneously stepping in for their lab assistant when he could not attend due to sickness.
Additionally the authors would like to thank the open-source community for providing them with the software tools and techniques used for data analysis in the form of R.
}

\end{document}
