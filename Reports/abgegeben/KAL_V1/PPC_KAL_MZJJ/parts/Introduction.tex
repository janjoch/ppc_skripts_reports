
\section{Introduction}

% Certain heat effects can be observed during chemical transformations, phase transitions or while solving or mixing different substances.

Certain heat effects can be observed during chemical transformations, phase transitions, dissolution or mixing of substances. 
The general rule is that in a closed system such as the calorimeter any change in the internal system energy $U$ only occurs through the exchange of heat $\delta Q$ or work $\delta W$ (here: work necessary to change the volume of the system against external pressure). 

If the volume remains constant, this results in the relationship 

\begin{equation} \label{eq:1} %\tag{1}
    dU = \delta Q
\end{equation} 
At constant pressure the enthalpy change $dH$  is also equal to the amount of heat exchanged: 

\begin{equation} \label{eq:2} %\tag{2}
    dH = \delta Q
\end{equation} 
If adiabatic conditions apply as well, the system's enthalpy remains the same and $dH = 0$ because no heat is exchanged. Within the system, thermal and chemical processes contribute to enthalpy changes:
\begin{equation} \label{eq:3} %\tag{3}
    dH = dH_{\mathrm{thermal}} + dH_{\mathrm{chemical}}
\end{equation}
If a change in temperature  $dT$  occurs within a system at constant pressure where no alteration to the substances' chemical compositions can be observed, the following relation can be established:
\begin{equation} \label{eq:4} %\tag{4}
    dH = dH_{\mathrm{thermal}} = C_p dT
\end{equation}
where the proportionality constant $C_p$ is called heat capacity of the system. 
If the materials' chemical compositions change at constant temperature and constant pressure, the following applies: 
\begin{equation} \label{eq:5} %\tag{5}
    dH = dH_{\mathrm{chemical}} = \Delta _rH dn
\end{equation}
where $\Delta _rH$ is called enthalpy of reaction and $dn$ refers to the amount $n$ of chemically converted substance.

Overall, under isobaric conditions the following applies: 
\begin{equation} \label{eq:6} %\tag{6}
    Q = C_p \Delta T + \Delta _rH \Delta n
\end{equation}
assuming the change in temperature and chemical composition are fairly minor and therefore considering $C_p$ and $\Delta _rH$ to be constant. 

$C_p$ of a given system is the sum of all its components heat capacities and can be determined via calibration by supplying a known amount of heat $Q = UIt$, where $U$ is the operating voltage and $I$ the operating current of the heating system used and $t$ is the duration of the heatflow to the system. 

$\Delta _rH$ can be found by thermically isolating the system and measuring $\Delta T$ during the respective reaction or process.


\newpage


\iffalse{
Welche formeln wurden verwendet:
 
Berechnung wärmekapazität kalorimeter:
    CpK = Cp - Cpw mit Cp= Q/DeltaT= UIt/DeltaT (wärmekapazität des gesamtsystems) und mw*cp_spez_w (wärmekapazität deionisiertes wasser) 

Berechnung spezifische Wärmekapazität der ethanol wasser msichungen: 
    cp_spez_eth = CpEth/mEth mit CpEth = C'p - CpK und C'p = UIt/DeltaT (gesamtwärmek. wie oben ' nur um zu kennzeichnen, dass anders als H2O) 

Berechnung molare Lösungsenthalpie ammonium nitrat
    Lösungsenth.=

Berechnung 

}